\documentclass{article}
\usepackage{graphicx}

\title{A Study of Infection Segmentation in X-ray images of COVID 19 patients}
\author{Nguyen Xuan Tung - BI12-479}
\date{\today}

\begin{document}

\maketitle
\section{Introduction}
% Write your introduction here
For the study, I built an U-Net model which is combined with Residual Block in the encoder part 
to generate the segmentation masks of the infected areas in X-ray images of COVID 19 patients.
This model was trained with 2 different loss functions: binary cross-entropy (BCE) and a combination
of BCE and Intersection over Union (BCE + IoU). 

\section{Data Analysis}
The dataset used in this study is the COVID-QU-Ex Dataset. It contains 2913 X-ray images of COVID 19 patients.
For data preprocessing, I resized all the images to 256x256 pixels in 1 channel and 
normalized them to the range [0, 1].

\section{Model}
Instead of ultilizing the original U-Net model like in the previous study, I added Residual Block to the encoder 
part of the U-Net model, which is expected to improve the ability of extracting features more effectively.

\subsection{U-Net Architecture with Residual Block}
This U-Net model is a convolutional neural network for fast and precise segmentation of images.
The architecture contains two paths: the contraction path (encoder) and the symmetric expanding path (decoder).
At the encoder part, the Residual Block is adapted in the down-sampling path in order to mitigate the vanishing gradient problem,
whereas the decoder part is the same as the original model.


\subsection{Loss Functions}
I used 2 loss functions to train the U-Net model: binary cross-entropy (BCE) and a combination of BCE and Intersection over Union (BCE + IoU).
For the combined loss function, I weighted the BCE and IoU loss with a hyperparameter $\alpha$. The following equation shows the combined loss function:
% combined_loss = self.alpha * self.bce(inputs, targets) + (1 - self.alpha) * self.iou(inputs, targets)
\begin{equation}
    BCE\_IoU = \alpha * BCE + (1 - \alpha) * IoU
\end{equation}

\section{Results}
\subsection{For detecting the infected segmentation masks}
Observing the loss values of the U-Net model training in 100 epochs with different loss functions, the model 
with only BCE loss seems to fit the X-ray dataset better than using combined model with BCE and IoU loss.
The results demonsrated that the combination loss function were not suitable for this task.
% Plot 3 pictures of loss values in rows
\begin{table}[h]
    \centering
    \begin{tabular}{cc}
        \includegraphics[width=0.4\textwidth]{bce.png} & 
        \includegraphics[width=0.4\textwidth]{bce_iou.png} \\
        (a) BCE loss & (b) BCE + IoU loss
    \end{tabular}
    \caption{Loss values of the U-Net model with Residual Block by different loss functions}
\end{table}

% Table of MAE and IoU values of 3 models
\begin{table}[h]
    \centering
    \begin{tabular}{|c|c|c|}
        \hline
        Loss function & MAE & IoU \\
        \hline
        BCE & 16.321 & 0.138 \\
        BCE + IoU & 102168.546 & 0.138 \\
        \hline
    \end{tabular}
    \caption{MAE and IoU values of the model with different loss functions on the test dataset}
\end{table}


Besides the evalution metrics, I also visualized the Infected Segmentation of the 181st, 182nd and 184th images of the test dataset
inferenced by the U-Net model with different loss functions. The results are shown in the figures 1 to 3.

% Plot some images of the head boundary
\begin{figure}[h]
    \centering
    \includegraphics[width=1.3\textwidth]{output 181.png}
    \caption{Infected Segmentation 181st image of the U-Net model with different loss functions}
\end{figure}

\begin{figure}[h]
    \centering
    \includegraphics[width=1.3\textwidth]{output 182.png}
    \caption{Infected Segmentation 182nd image of the U-Net model with different loss functions}
\end{figure}

\begin{figure}[h]
    \centering
    \includegraphics[width=1.3\textwidth]{output 184.png}
    \caption{Infected Segmentation 184th image of the U-Net model with different loss functions}
\end{figure}


\end{document}