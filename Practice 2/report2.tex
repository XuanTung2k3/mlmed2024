\documentclass{article}
\usepackage{graphicx}

\title{A Study of Automated measurement of fetal head circumference using 2D ultrasound images}
\author{Nguyen Xuan Tung - BI12-479}
\date{\today}

\begin{document}

\maketitle
\section{Introduction}
% Write your introduction here
In this assignment, I tried to implement some U-Net models with different loss functions to 
detect the head boundary in 2D ultrasound images, and then measure the head circumference.

\section{Data Analysis}
The dataset used in this study is the HC18 dataset. The training dataset consists of 999 2D ultrasound images of fetal head 
and the corresponding ground truth masks, whereas the test dataset consists of 335 2D ultrasound images of fetal head with no labels.
For training the model, 80\% of the training dataset will be used for training, 10\% for validation, and 10\% for testing to  
evaluate the model.\\ 

\section{Model}
Instead of building a model for segmentation, I used the U-Net model with different loss functions 
to generate new images of only the head boundary like the ground truth masks. After that, I utilized
some image processing techniques to determine the contour of the head boundary and measure the head circumference.

\subsection{U-Net Architecture}
The U-Net model is a convolutional neural network for fast and precise segmentation of images.
The architecture contains two paths: the contraction path (encoder) and the symmetric expanding path (decoder).
The encoder is used to capture the context of the input image, while the decoder is used to enable precise localization.

\begin{figure}[h]
    \centering
    \includegraphics[width=1.3\textwidth]{u-net.jpg}
    \caption{U-Net Architecture}
\end{figure}

\subsection{Loss Functions}
I used three loss functions to train the U-Net model: mean squared error (MSE), binary cross-entropy (BCE), 
and binary cross entropy combined with Intersection over Union (BCE + IoU). 
Firstly, the MSE loss function is used to measure the average of the squares of the errors between the predicted and ground truth masks. 
Next, I used the BCE loss function to measure the difference between the predicted and ground truth masks. 
Finally, a combination of BCE and IoU loss functions is an attempt to improve the performance of the model.
The combination loss function is defined as follows:
% combined_loss = self.alpha * self.bce(inputs, targets) + (1 - self.alpha) * self.iou(inputs, targets)
\begin{equation}
    MSE\_IoU = \alpha * BCE + (1 - \alpha) * IoU
\end{equation}

\section{Results}
\subsection{For detecting the head boundary}
Observing the loss values of the U-Net model training in 20 epochs with different loss functions, the model 
with MSE loss has a decrease in loss value over time, while other cases didn't have significant changes in loss values.
% Plot 3 pictures of loss values in rows
\begin{table}[h]
    \centering
    \begin{tabular}{ccc}
        \includegraphics[width=0.4\textwidth]{mse_loss.png} & 
        \includegraphics[width=0.4\textwidth]{BCE_loss.png} & 
        \includegraphics[width=0.4\textwidth]{bce_iou_loss.png} \\
        (a) MSE loss & (b) BCE loss & (c) MSE + IoU loss
    \end{tabular}
    \caption{Loss values of the U-Net model with different loss functions}
\end{table}

% Table of MAE and IoU values of 3 models
\begin{table}[h]
    \centering
    \begin{tabular}{|c|c|c|}
        \hline
        Loss function & MAE & IoU \\
        \hline
        MSE & 0.0108 & 0.0226 \\
        BCE & 38.0359 & 0.2237 \\
        BCE + IoU & 11.6979 & 0.0224 \\
        \hline
    \end{tabular}
    \caption{MAE and IoU values of the U-Net model with different loss functions for test dataset}
\end{table}



\noindent As can be seen from the table 2, the U-Net model with MSE loss function has the lowest MAE value,
while the model with combination loss function has the best IoU value.

Besides the evalution metrics, I also visualized the head boundary of the 200th, 201st, and 202nd images of the test dataset
inferenced by the U-Net model with different loss functions. The results are shown in the figures 2, 3 and 4.

% Plot some images of the head boundary
\begin{figure}[h]
    \centering
    \includegraphics[width=1.3\textwidth]{inference test 1.png}
    \caption{Head boundary 200th image of the U-Net model with different loss functions}
\end{figure}

\begin{figure}[h]
    \centering
    \includegraphics[width=1.3\textwidth]{inference test 201.png}
    \caption{Head boundary 201st image of the U-Net model with different loss functions}
\end{figure}

\begin{figure}[h]
    \centering
    \includegraphics[width=1.3\textwidth]{inference test 202.png}
    \caption{Head boundary 202nd image of the U-Net model with different loss functions}
\end{figure}



\end{document}